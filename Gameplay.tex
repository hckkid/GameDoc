\chapter{Gameplay}
\section{Gameplay Modes}
Below is a list of Gameplay modes for our Game
\begin{itemize}
\item The Walking Legend                    % for caimpaign mode
\item In Company of Heroes                  % for rts mode
\item Scars of Fate                         % for short caimpaign mode
\end{itemize}
\subsection{The Walking Legend}
This is basically going to be a Campaign Mode. In this mode one will have to select from either Citizen or Terrorist or Counter-Terrorist. In campaign mode our main focus will be on RPG\footnote{RPG - Role Playing Game}. This mode will have a storyline for corresponding character selected with some interlink between stories of other characters. So we will have a whole storyline Graph with three different starting points over which one can proceed forward in time. So far so familiar so what's the difference. The concept of storyline graph is used in many games such as Mass Effect, Dragon Age etc. But lets us know introduce something new consider the real time situation is the two games stated above. If we had real situation then it would be more like enemies would have been also growing better as we are progressing through the game. In the two stated Games there is very little race against time. In these games they intelligence of opponents grows with your levelling up not the time and also they do level up but do not create exclusion. Here the exclusion we are talking of isn't permanent but more like in RTS\footnote{RTS - Real Time Strategy} games. As in RTS consider a situation based on territories. Now assume you are in your territory and so is your opponent. Now there are two unclaimed territories between you and your opponent. So now as you proceed with capturing one these simultaneously will your opponent for other. Battle for Wesnoth\footnote{Battle for Wesnoth is a free and open source game one can easily find it on ubuntu marketplace} a.k.a BOW which is a turn based strategy game shows it clearly. So what we are trying to do is to make this kind of exclusion pretty close to timeline graph of game to create a all new gameplay experience. This whole new experience is the reason why we call our game a concept game which we are developing both to learn game development in depth and to test out our idea. The Walking Legend mode is estimated go on around 7-8 hours of play time. The timeline graph \& Storyline Plots will be provided in separate chapter
\subsection{In Companies of Heroes}
This mode is intended for Multiplayer gaming or against bots\footnote{bots are term for AI controlled players throughout this document and also AI in this whole document stands for Artificial Intelligence}. There would be very little of RPG flavor to it and it would be mostly an RTS version of our game. Each game would consist of a small terrain and every game would have almost same basic objectives independent of terrain. How to achieve them will be highly influenced by terrain. Also objectives would be different for different classes that user picks.
\subsection{Scars of Fate}
This mode is for providing more story links to main storyline time graph. Its kind of extension and for DLC\footnote{DLC :- Downloadable Content}. The levels/campaigns in this mode would be small but would be covering interesting stuff regarding storyline and of course there would be always a surprise element awaiting you to explore it. So its highly suggested to play these levels/campaigns only when you have finished The Walking Legend campaign of corresponding class and also to not miss playing these short stories.